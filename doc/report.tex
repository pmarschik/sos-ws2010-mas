\title{Machine Learning \\ \bigskip WS2010/11\\ \bigskip Exercise 2}
\author{
        Benjamin Bachhuber \\
                (1028430)\\
                \and
        Patrick Marschik\\
        (0625039)
}
\date{\today}

\documentclass[a4paper,12pt]{article}
\usepackage{verbatim}
\usepackage{graphicx}
\usepackage{color}
\usepackage{array}
\usepackage{xspace}
\usepackage{acronym}
\usepackage{rotating}

\definecolor{LinkColor}{rgb}{0,0,0.7}

\usepackage[
bookmarks=true, % PDF bookmarks allowed. NB! The level depth of bookmarks is the same as in the TOC.
unicode=true, % PDF bookmarks in Unicode.
bookmarksnumbered=true, % Section numbers in PDF bookmarks.
bookmarksopenlevel=1, % The open level in PDF bookmarks.
hyperindex=true, % Hyperlinked index.
plainpages=false, % Name arabic and roman page numbers differently.
colorlinks=true, % Links are marked as coloured text, not coloured box.
linkcolor=linkc, % The colour for in-document links (e.g. in the table of contents).
citecolor = citec, % The colour for bibliographic citations.
urlcolor=urlc, % The colour for hyperlinks to the Net.
pdfpagelayout=OneColumn % Continuous page scrolling.
]{hyperref}
\hypersetup{
  colorlinks=true,
  linkcolor=LinkColor,
  citecolor=LinkColor,
  filecolor=LinkColor,
  menucolor=LinkColor,
  pagecolor=LinkColor,
  urlcolor=LinkColor
}

\usepackage{amsmath}
\usepackage{amsthm}
\usepackage{amsfonts}

\usepackage{tocloft}

\newcommand{\dss}{soybean-small\xspace}
\newcommand{\dsn}{nursery\xspace}
\newcommand{\dsa}{arrhythmia\xspace}
\newcommand{\linsearch}{LinearNNSearch\xspace}

\acrodef{WEKA}{Waikato Environment for Knowledge Analysis}

\begin{document}
\maketitle

\newcommand{\fittable}[1]{\begin{center}
\resizebox{\textwidth}{!}{#1}\end{center}
}

\newcommand{\img}[4]{
 \begin{figure}[!htp]
\centering
  \includegraphics[width=#1\textwidth]{#2}
  \caption{\emph{#3}}
  \label{#4}
 \end{figure}
}
\newcommand{\imgit}[4]{
\\[\intextsep]
\begin{minipage}{\linewidth}
  \centering%
  \includegraphics[width=#1\textwidth,clip=]{#2}%
  \figcaption{\emph{#3}}%
  \label{#4}%
\end{minipage}
\\[\intextsep]
}

\newcommand{\imgitX}[2]{
  {\centering
\fbox{
\begin{minipage}{#1\textwidth}
  \vskip .5em
  \centering
  \includegraphics[width=#1\textwidth]{#2}
  \vskip .5em
\end{minipage}
}
  }
}


\newpage

\tableofcontents

\section{Introduction}

The prisoner's dilemma is one of the fundamental problems in game theory.\\
There are two prisoner's. In the prisoners dilemma each of them has to decide whether to cooperate with an opponent, or defect. Both prisoners make a choice and then their decisions are revealed. 

They receive a payoff according to the following matrix (from the point of view of prisoner A). 

\begin{table*}[h]
	\centering
		\begin{tabular}{| l | r | r |}
		\hline
			Action of A / Action of B & Comply & Defect \\
			Comply                    & 3 & 0\\
			Defect                    & 5 & 1\\
			\hline
		\end{tabular}  
\end{table*}

Since the strategies for a game with one turn or with a deterministic amount of games is fairly predictable, the so called \textit{Iterate Prisoner's Dilemma} was used. In this special kind of Prisoner's Dilemma none of the prisoner's knows how many rounds are played.

\section{Technical Description}

One of the most important decisions is which agents are needed and what their main tasks are. The following use case diagramm helped was very useful in this process of decision making:

%%%% Use Case Diagramm hier

The use case diagram shows that two agent classes are needed: PrisonerAgent and GameMasterAgent.


The Agents require to communicate with eachother on several channels. There are different forms of communication which are required:


\begin{itemize}
	\item The GameMasterAgent has to be able to ask the the PrisonerAgents if they are guilty
	\item The PrisonerAgents has to be able to answer if they are guilty
	\item A PrisonerAgent has to be able to request notifications about the recent game state from the GameMasterAgent
	\item The GameMaster has to be able to inform PrisonerAgents which requested notifications about changes in the game state.
\end{itemize}

This tasks basically require two different interaction protocols:


\begin{description}
	\item[ArchieveREInitiator \& ArchieveREResponder]\hfill \\
	 This protocol is used to to enable the Agents to ask and answer the question of guiltiness.
	\item[SubscriptionInitiator \& SubscriptionResponder] \hfill \\
	Enables the PrisonerAgent to receive notification about the newest game state.
\end{description}


\img{1}{chart1_Part2.pdf}{}{}
\img{1}{chart1_Part3.pdf}{}{}

\img{1}{chart1_Part1.pdf}{State Chart}{}

\section{Conclusion}
The prisoner's dilemma is a good example to show the strength and weaknesses of Multi Agent Systems.
Using Multi Agent Systems for small projects seems like an overhead but in large projects Multi Agents System are one of the most powerful architecture available.


\newpage






%#\bibliographystyle{abbrv}
%#\bibliography{main}

\end{document}
This is never printed

