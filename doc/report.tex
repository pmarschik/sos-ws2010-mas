\title{Machine Learning \\ \bigskip WS2010/11\\ \bigskip Exercise 2}
\author{
        Benjamin Bachhuber \\
                (1028430)\\
                \and
        Patrick Marschik\\
        (0625039)
}
\date{\today}

\documentclass[a4paper,12pt]{article}
\usepackage{verbatim}
\usepackage{graphicx}
\usepackage{color}
\usepackage{array}
\usepackage{xspace}
\usepackage{acronym}
\usepackage{rotating}

\definecolor{LinkColor}{rgb}{0,0,0.7}

\usepackage[
bookmarks=true, % PDF bookmarks allowed. NB! The level depth of bookmarks is the same as in the TOC.
unicode=true, % PDF bookmarks in Unicode.
bookmarksnumbered=true, % Section numbers in PDF bookmarks.
bookmarksopenlevel=1, % The open level in PDF bookmarks.
hyperindex=true, % Hyperlinked index.
plainpages=false, % Name arabic and roman page numbers differently.
colorlinks=true, % Links are marked as coloured text, not coloured box.
linkcolor=linkc, % The colour for in-document links (e.g. in the table of contents).
citecolor = citec, % The colour for bibliographic citations.
urlcolor=urlc, % The colour for hyperlinks to the Net.
pdfpagelayout=OneColumn % Continuous page scrolling.
]{hyperref}
\hypersetup{
  colorlinks=true,
  linkcolor=LinkColor,
  citecolor=LinkColor,
  filecolor=LinkColor,
  menucolor=LinkColor,
  pagecolor=LinkColor,
  urlcolor=LinkColor
}

\usepackage{amsmath}
\usepackage{amsthm}
\usepackage{amsfonts}

\usepackage{tocloft}

\newcommand{\dss}{soybean-small\xspace}
\newcommand{\dsn}{nursery\xspace}
\newcommand{\dsa}{arrhythmia\xspace}
\newcommand{\linsearch}{LinearNNSearch\xspace}

\acrodef{WEKA}{Waikato Environment for Knowledge Analysis}

\begin{document}
\maketitle

\newcommand{\fittable}[1]{\begin{center}
\resizebox{\textwidth}{!}{#1}\end{center}
}

\newcommand{\img}[4]{
 \begin{figure}[!htp]
\centering
  \includegraphics[width=#1\textwidth]{#2}
  \caption{\emph{#3}}
  \label{#4}
 \end{figure}
}
\newcommand{\imgit}[4]{
\\[\intextsep]
\begin{minipage}{\linewidth}
  \centering%
  \includegraphics[width=#1\textwidth,clip=]{#2}%
  \figcaption{\emph{#3}}%
  \label{#4}%
\end{minipage}
\\[\intextsep]
}

\newcommand{\imgitX}[2]{
  {\centering
\fbox{
\begin{minipage}{#1\textwidth}
  \vskip .5em
  \centering
  \includegraphics[width=#1\textwidth]{#2}
  \vskip .5em
\end{minipage}
}
  }
}


\newpage

\tableofcontents

\section{Introduction}

The prisoner's dilemma is one of the fundamental problems in game theory.



\img{1}{chart1_Part1.pdf}{State Chart}{}
\img{1}{chart1_Part2.pdf}{}{}
\img{1}{chart1_Part3.pdf}{}{}

\section{Conclusion}
The prisoner's dilemma is a good example to show the strength and weaknesses of Multi Agent Systems.
Using Multi Agent Systems for small projects seems like an overhead but in large projects Multi Agents System are one of the most powerful architecture available.


\newpage






%#\bibliographystyle{abbrv}
%#\bibliography{main}

\end{document}
This is never printed

